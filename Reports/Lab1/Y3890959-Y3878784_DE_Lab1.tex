\documentclass[11pt]{report}
\usepackage[utf8]{inputenc}
\usepackage[margin=2.0cm]{geometry}
\usepackage{fancyhdr}
\usepackage{xcolor}
\usepackage{minted}
\usepackage{graphicx}

\title{Digital Engineering\\Lab 1}
\author{Y3890959\\Y3878784}
\date{24th January 2023}

\pagestyle{fancy}
\fancyhead{}
\setlength{\headheight}{14pt}
\fancyhead[L]{Lab 1}
\fancyhead[R]{Y3890959, Y3878784}
\fancyfoot{}
\fancyfoot[L]{Digital Engineering}
\fancyfoot[R]{\thepage}

\makeatletter
\let\ps@plain\ps@fancy 
\makeatother

\setminted {
    fontsize=\footnotesize,
    frame=single,
}

\begin{document}

\maketitle

\chapter*{Task A: Debouncer implementation and simulation by parameterization}

\section*{VHDL Code}

\subsection*{Top-Level Entity}
\inputminted{vhdl}{../../Lab1/Lab1.srcs/sources_1/imports/new/fibonacci_8bit_sequence.vhd}

\subsection*{Efficient (Updated) Debouncer Entity}
\inputminted{vhdl}{../../Lab1/Lab1.srcs/sources_1/new/efficient_debouncer.vhd}

\subsection*{Counter Entity}
\inputminted{vhdl}{../../Lab1/Lab1.srcs/sources_1/new/parameterizable_counter.vhd}

\subsection*{ROM Entity}
\inputminted{vhdl}{../../Lab1/Lab1.srcs/sources_1/imports/new/fibonacci_8bit_async_read_rom.vhd}



\section*{Testbenches and Simulation}

\subsection*{Testbech VHDL Code}
\inputminted{vhdl}{../../Lab1/Lab1.srcs/sim_1/imports/new/fibonacci_8bit_sequence_tb.vhd}



\subsection*{Testbench Waveforms}

\subsubsection*{Entire Sequence}
\begin{figure}[H]
    \includegraphics[width=\columnwidth]{Reports/Lab1/Waveforms/01_entire-sequence.png}
\end{figure}
The above waveform screenshot shows the entire sequence from time 0 to output 1 after a roll-over. This waveform is intended to verify that the overall output is correct.

\subsubsection*{Test 1, Waveform 1 - Initial reset and counting}
\begin{figure}[H]
    \includegraphics[width=\columnwidth]{Reports/Lab1/Waveforms/02_initial-reset-and-counting.png}
\end{figure}
The above waveform shows the initial global reset, verifying a successful activation of both the "reset" and "count" debouncers and a successful reset and increment of the overall circuit. 

\subsubsection*{Test 1, Waveform 2 - Counting up}
\begin{figure}[H]
    \includegraphics[width=\columnwidth]{Reports/Lab1/Waveforms/03_counting-up.png}
\end{figure}
This waveform (a continuation of Test 1, Waveform 1) verifies that the debouncer is activating and the counter increments as designed, the Fibonacci sequence is shown (1,2,3,5,8).

\subsubsection*{Test 1, Waveform 3 - Roll-over and count up}
\begin{figure}[H]
    \includegraphics[width=\columnwidth]{Reports/Lab1/Waveforms/04_roll-over-and-count.png}
\end{figure}
This waveform verifies that the maximum Fibonacci sequence value is reached, and the sequence rolls over to 0 without a reset signal being generated by the user and can resume counting up.

\subsubsection*{Test 2 - Quick reset pulse}
\begin{figure}[H]
    \includegraphics[width=\columnwidth]{Reports/Lab1/Waveforms/05_quick-reset-pulse.png}
\end{figure}
This waveform verifies that the debouncer is functioning as intended by showing the parameterizable counter on board the debouncer incrementing when the reset signal is high but resetting as soon as it goes low before the 'LIMIT' is reached, thus invalidating the reset input.

\subsubsection*{Test 3 - Longer reset pulse}
\begin{figure}[H]
    \includegraphics[width=\columnwidth]{Reports/Lab1/Waveforms/06_longer-reset-pulse.png}
\end{figure}
This waveform verifies that the debouncer successfully activates with a longer reset pulse of a valid clock duration. The counter in the debouncer increments with each clock cycle while the reset is high, and outputs a high debounced output once 'LIMIT' is reached.

\subsubsection*{Test 4 - Resume after mid-operation reset}
\begin{figure}[H]
    \includegraphics[width=\columnwidth]{Reports/Lab1/Waveforms/07_resume-after-mid-operation-reset.png}
\end{figure}
This waveform verifies that the circuit successfully resumes counting after a mid-operation reset following Test 3.

\subsubsection*{Test 6 - Holding reset and count}
\begin{figure}[H]
    \includegraphics[width=\columnwidth]{Reports/Lab1/Waveforms/08_holding-reset-and-count.png}
\end{figure}

\section*{Circuit Analysis and Synthesis}

\subsection*{RTL Componenet Statistics}

\subsection*{RTL Hierarchical Component Statistics}

\subsection*{Schematics}

\subsubsection*{RTL Top Level (ROM Expanded)}

\subsubsection*{RTL Debouncer}

\subsubsection*{Synthesis Top Level (ROM Expanded)}

\end{document}
