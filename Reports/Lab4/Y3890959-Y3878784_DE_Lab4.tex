\documentclass[11pt]{report}
\usepackage[utf8]{inputenc}
\usepackage[margin=2.0cm]{geometry}
\usepackage{fancyhdr}
\usepackage{xcolor}
\usepackage{minted}
\usepackage{graphicx}
\usepackage[parfill]{parskip}
\usepackage{tabularx,colortbl}

\title{Digital Engineering\\Lab 4}
\author{Y3890959\\Y3878784}
\date{14th February 2023}

\pagestyle{fancy}
\fancyhead{}
\setlength{\headheight}{14pt}
\fancyhead[L]{Lab 4}
\fancyhead[R]{Y3890959, Y3878784}
\fancyfoot{}
\fancyfoot[L]{Digital Engineering}
\fancyfoot[R]{\thepage}

\makeatletter
\let\ps@plain\ps@fancy 
\makeatother

\setminted {
    fontsize=\footnotesize,
    frame=single,
}

\begin{document}

\maketitle

\chapter*{Task A: Test Pattern Generation}

\section*{3.1.1 Gate Level Fault Collapsing}
Key: The inputs are X and Y. The output is Z. The outputs that don't match the desired have been highlighted yellow to show that the fault can be detected.

\subsection*{AND Gate}
\begin{tabular}{ |c|c|c|c|c|c|c|c| }
\hline
INPUT XY & DESIRED Z & Xs-a-0 & Xs-a-1 & Ys-a-0 & Ys-a-1 & Zs-a-0 & Zs-a-1
\\ 
\hline
\hline
00 & 0 & 0 & 0 & 0 & 0 & 0 & \cellcolor{yellow!50}1
\\  
\hline
01 & 0 & 0 & \cellcolor{yellow!50}1 & 0 & 0 & 0 & \cellcolor{yellow!50}1
\\
\hline
10 & 0 & 0 & 0 & 0 & \cellcolor{yellow!50}1 & 0 & \cellcolor{yellow!50}1
\\
\hline
11 & 1 & \cellcolor{yellow!50}0 & 1 & \cellcolor{yellow!50}0 & 1 & \cellcolor{yellow!50}0 & 1
\\
\hline
\end{tabular}

\begin{tabular}{ |c|c|c| }
\hline
Equivalent & Dominant & Dominated
\\ 
\hline
\hline
Xs-a-0 & Xs-a-1 & Zs-a-1
\\  
\hline
Ys-a-0 & Ys-a-1 & 
\\
\hline
Zs-a-0 & & 
\\
\hline
\end{tabular}



\subsection*{OR Gate}
\begin{tabular}{ |c|c|c|c|c|c|c|c| }
\hline
INPUT XY & DESIRED Z & Xs-a-0 & Xs-a-1 & Ys-a-0 & Ys-a-1 & Zs-a-0 & Zs-a-1
\\ 
\hline
\hline
00 & 0 & 0 & \cellcolor{yellow!50}1 & 0 & \cellcolor{yellow!50}1 & 0 & \cellcolor{yellow!50}1
\\  
\hline
01 & 1 & 1 & 1 & \cellcolor{yellow!50}0 & 1 & \cellcolor{yellow!50}0 & 1
\\
\hline
10 & 1 & \cellcolor{yellow!50}0 & 1 & 1 & 1 & \cellcolor{yellow!50}0 & 1
\\
\hline
11 & 1 & 1 & 1 & 1 & 1 & \cellcolor{yellow!50}0 & 1
\\
\hline
\end{tabular}

\begin{tabular}{ |c|c|c| }
\hline
Equivalent & Dominant & Dominated
\\ 
\hline
\hline
Xs-a-1 & Xs-a-0 & Zs-a-0
\\  
\hline
Ys-a-1 & Ys-a-0 & 
\\
\hline
Zs-a-1 & & 
\\
\hline
\end{tabular}



\subsection*{NOR Gate}
\begin{tabular}{ |c|c|c|c|c|c|c|c| }
\hline
INPUT XY & DESIRED Z & Xs-a-0 & Xs-a-1 & Ys-a-0 & Ys-a-1 & Zs-a-0 & Zs-a-1
\\ 
\hline
\hline
00 & 1 & 1 & \cellcolor{yellow!50}0 & 1 & \cellcolor{yellow!50}0 & \cellcolor{yellow!50}0 & 1
\\  
\hline
01 & 0 & 0 & 0 & \cellcolor{yellow!50}1 & 0 & 0 & \cellcolor{yellow!50}1
\\
\hline
10 & 0 & \cellcolor{yellow!50}1 & 0 & 0 & 0 & 0 & \cellcolor{yellow!50}1
\\
\hline
11 & 0 & 0 & 0 & 0 & 0 & 0 & \cellcolor{yellow!50}1
\\
\hline
\end{tabular}

\begin{tabular}{ |c|c|c| }
\hline
Equivalent & Dominant & Dominated
\\ 
\hline
\hline
Xs-a-1 & Xs-a-0 & Zs-a-1
\\  
\hline
Ys-a-1 & Ys-a-0 & 
\\
\hline
Zs-a-0 & & 
\\
\hline
\end{tabular}



\section*{3.1.2 Circuit Fault Collapsing}
First considering the AND gate with inputs A and B, the equivalent faults
are As-a-0, Bs-a-0, and Gs-a-0. We can eliminate Bs-a-0 and Gs-a-0 as
only As-a-0 is required to test the rest. Gs-a-1 is dominated by As-a-1
and Bs-a-1, therefore Gs-a-1 can also be eliminated.

Now considering the NOR gate with inputs G and I0, the equivalent faults
are Gs-a-1, I0s-a-1, and Js-a-0. Gs-a-1 has already been eliminated, we can
further eliminate I0s-a-1 and Js-a-0. Js-a-1 is being dominated by Gs-a-0 and
I0s-a-0, therefore we can eliminate Js-a-1 too.

Next considering the OR gate with inputs J and K, the equivalent faults
are Js-a-1, Ks-a-1, and Ls-a-1. Js-a-1 has already been eliminated, we can
further eliminate both Ks-a-1 and Ls-a-1. Ls-a-0 is dominated by Js-a-0 and
Ks-a-0, therefore we can also eliminate Ls-a-0.

Let's consider the AND gate with inputs C and D, the equivalent faults are
Cs-a-0, Ds-a-0, and Hs-a-0. We can eliminate Ds-a-0 and Hs-a-0 as only Cs-a-0
is needed to test the other two. Hs-a-1 is dominated by Cs-a-1 and Ds-a-1, so
we can further eliminate Hs-a-1.

Now considering the OR gate with inputs H and E, the equivalent faults are
Hs-a-1, Es-a-1, and Is-a-1. Hs-a-1 has already been eliminated, we can further
eliminate both Es-a-1 and Is-a-1. Is-a-0 is being dominated by Hs-a-0 and
Es-a-0, hence we can eliminate Is-a-0.

Next considering the OR gate with inputs I1 and F, the equivalent faults are
I1s-a-1, Fs-a-1, and Ks-a-1. Ks-a-1 has already been eliminated, we can also
eliminate I1s-a-1 and Fs-a-1. Ks-a-0 is dominated by I1s-a-0 and Fs-a-0, we
can therefore eliminate Ks-a-0.

Finally, considering the OR gate with inputs J and K, the equivalent faults
are Js-a-1, Ks-a-1, and Ls-a-1. All of these have already been eliminated.
Ls-a-0 is dominated by Js-a-0 and Ks-a-0, Ls-a-0 has already been eliminated.

After this process, we are left with the following 10 faults:
\begin{itemize}
    \item Node As-a-0
    \item Node As-a-1
    \item Node I0s-a-0
    \item Node Ds-a-1
    \item Node Es-a-0
    \item Node I1s-a-0
    \item Node Bs-a-1
    \item Node Cs-a-0
    \item Node Cs-a-1
    \item Node Fs-a-0
\end{itemize}


\section*{3.1.3 The D-Algorithm}

\begin{tabular}{ |c||c|c|c|c|c|c|c|c|c|c|c|c|c|c|}
\hline
\bf As-a-0 & \bf A & \bf B & \bf C & \bf D & \bf E & \bf F & G & H & I & I\textsubscript{0} & I\textsubscript{1} & J & K & \bf L \\
\hline
\hline
Step 1 & D & 1 & & & & & D & & & & & & & \\
\hline
Step 2 & D & 1 & & & & & D & & & 0 & & D' & & \\
\hline
Step 3 & D & 1 & & & & & D & & & 0 & & D' & 0 & D' \\
\hline
Step 4 & D & 1 & & & & 0 & D & & 0 & 0 & 0 & D' & 0 & D' \\
\hline
Step 5 & D & 1 & & & 0 & 0 & D & 0 & 0 & 0 & 0 & D' & 0 & D' \\
\hline
Step 6 & D & 1 & X & 0 & 0 & 0 & D & 0 & 0 & 0 & 0 & D' & 0 & D' \\
\hline
\end{tabular}


\begin{tabular}{ |c||c|c|c|c|c|c|c|c|c|c|c|c|c|c| }
\hline
\bf As-a-1 & \bf A & \bf B & \bf C & \bf D & \bf E & \bf F & G & H & I & I\textsubscript{0} & I\textsubscript{1} & J & K & \bf L \\
\hline
\hline
Step 1 & D' & 1 & & & & & D' & & & & & & & \\
\hline
Step 2 & D' & 1 & & & & & D' & & 0 & 0 & 0 & D & & \\
\hline
Step 3 & D' & 1 & & & & & D' & & 0 & 0 & 0 & D & 0 & D \\
\hline
Step 4 & D' & 1 & & & & 0 & D' & & 0 & 0 & 0 & D & 0 & D \\
\hline
Step 5 & D' & 1 & & & 0 & 0 & D' & 0 & 0 & 0 & 0 & D & 0 & D \\
\hline
Step 6 & D' & 1 & X & 0 & 0 & 0 & D' & 0 & 0 & 0 & 0 & D & 0 & D \\
\hline
\end{tabular}



\begin{tabular}{ |c||c|c|c|c|c|c|c|c|c|c|c|c|c|c| }
\hline
\bf I\textsubscript{0}s-a-0 & \bf A & \bf B & \bf C & \bf D & \bf E & \bf F & G & H & I & I\textsubscript{0} & I\textsubscript{1} & J & K & \bf L \\
\hline
\hline
Step 1 & & & & & & & 0 & & 1 & D & 1 & D' & & \\
\hline
Step 2 & 0 & 0 & & & & & 0 & & 1 & D & 1 & D' & & \\
\hline
Step 3 & 0 & 0 & & & 1 & & 0 & 1 & 1 & D & 1 & D' & & \\
\hline
Step 4 & 0 & 0 & 1 & 1 & 1 & & 0 & 1 & 1 & D & 1 & D' & & \\
\hline
Step 5 & 0 & 0 & 1 & 1 & 1 & & 0 & 1 & 1 & D & 1 & D' & 0 & D' \\
\hline
Step 6 & X & 0 & 1 & 1 & 1 & & 0 & 1 & 1 & D & \cellcolor{yellow!50}1 & D' & \cellcolor{yellow!50}0 & D' \\
\hline
\end{tabular}

I\textsubscript{0}s-a-1 is an undetectable fault. The yellow highlighted cells show the reason for this, with I\textsubscript{1} at `1', K cannot be `0' as this is an `OR' gate.

\begin{tabular}{ |c||c|c|c|c|c|c|c|c|c|c|c|c|c|c| }
\hline
\bf Ds-a-1 & \bf A & \bf B & \bf C & \bf D & \bf E & \bf F & G & H & I & I\textsubscript{0} & I\textsubscript{1} & J & K & \bf L \\
\hline
\hline
Step 1 & & & 1 & D' & & & & D' & & & & & & \\
\hline
Step 2 & & & 1 & D' & 0 & & & D' & D' & D' & D' & & & \\
\hline
Step 3 & & & 1 & D' & 0 & 0 & & D' & D' & D' & D' & & D' & \\
\hline
Step 4 & & & 1 & D' & 0 & 0 & & D' & D' & D' & D' & 0 & D' & D' \\
\hline
Step 5 & & & 1 & D' & 0 & 0 & 1 & D' & D' & D' & D' & 0 & D' & D' \\
\hline
Step 6 & 1 & 1 & 1 & D' & 0 & 0 & 1 & D' & D' & D' & D' & 0 & D' & D' \\
\hline
\end{tabular}



\begin{tabular}{ |c||c|c|c|c|c|c|c|c|c|c|c|c|c|c| }
\hline
\bf Es-a-0 & \bf A & \bf B & \bf C & \bf D & \bf E & \bf F & G & H & I & I\textsubscript{0} & I\textsubscript{1} & J & K & \bf L \\
\hline
\hline
Step 1 & & & & & D & & & 0 & D & D & D & & & \\
\hline
Step 2 & & & 0 & 0 & D & & & 0 & D & D & D & & & \\
\hline
Step 3 & & & 0 & 0 & D & 0 & & 0 & D & D & D & & D & \\
\hline
Step 4 & & & 0 & 0 & D & 0 & & 0 & D & D & D & 0 & D & D \\
\hline
Step 5 & & & 0 & 0 & D & 0 & 1 & 0 & D & D & D & 0 & D & D \\
\hline
Step 6 & 1 & 1 & X & 0 & D & 0 & 1 & 0 & D & D & D & 0 & D & D \\
\hline
\end{tabular}



\begin{tabular}{ |c||c|c|c|c|c|c|c|c|c|c|c|c|c|c| }
\hline
\bf I\textsubscript{1}s-a-0 & \bf A & \bf B & \bf C & \bf D & \bf E & \bf F & G & H & I & I\textsubscript{0} & I\textsubscript{1} & J & K & \bf L \\
\hline
\hline
Step 1 & & & & & 0 & & & 1 & 1 & 1 & D & & & \\
\hline
Step 2 & & & 1 & 1 & 0 & & & 1 & 1 & 1 & D & & & \\
\hline
Step 3 & & & 1 & 1 & 0 & 0 & & 1 & 1 & 1 & D & & D & \\
\hline
Step 4 & & & 1 & 1 & 0 & 0 & & 1 & 1 & 1 & D & 0 & D & D \\
\hline
Step 5 & X & X & 1 & 1 & 0 & 0 & X & 1 & 1 & 1 & D & 0 & D & D \\
\hline
\end{tabular}



\begin{tabular}{ |c||c|c|c|c|c|c|c|c|c|c|c|c|c|c| }
\hline
\bf Bs-a-1 & \bf A & \bf B & \bf C & \bf D & \bf E & \bf F & G & H & I & I\textsubscript{0} & I\textsubscript{1} & J & K & \bf L \\
\hline
\hline
Step 1 & 1 & D' & & & & & D' & & & & & & & \\
\hline
Step 2 & 1 & D' & & & & & D' & & 0 & 0 & 0 & D & & \\
\hline
Step 3 & 1 & D' & 0 & 0 & 0 & & D' & 0 & 0 & 0 & 0 & D & & \\
\hline
Step 4 & 1 & D' & 0 & 0 & 0 & & D' & 0 & 0 & 0 & 0 & D & 0 & D \\
\hline
Step 5 & 1 & D' & X & 0 & 0 & 0 & D' & 0 & 0 & 0 & 0 & D & 0 & D \\
\hline
\end{tabular}




\begin{tabular}{ |c||c|c|c|c|c|c|c|c|c|c|c|c|c|c| }
\hline
\bf Cs-a-0 & \bf A & \bf B & \bf C & \bf D & \bf E & \bf F & G & H & I & I\textsubscript{0} & I\textsubscript{1} & J & K & \bf L \\
\hline
\hline
Step 1 & & & D & 1 & & & & D & & & & & & \\
\hline
Step 2 & & & D & 1 & 0 & & & D & D & D & D & & & \\
\hline
Step 3 & & & D & 1 & 0 & 0 & & D & D & D & D & & D & \\
\hline
Step 4 & & & D & 1 & 0 & 0 & & D & D & D & D & 0 & D & D \\
\hline
Step 5 & & & D & 1 & 0 & 0 & 1 & D & D & D & D & 0 & D & D \\
\hline
Step 6 & 1 & 1 & D & 1 & 0 & 0 & 1 & D & D & D & D & 0 & D & D \\
\hline
\end{tabular}




\begin{tabular}{ |c||c|c|c|c|c|c|c|c|c|c|c|c|c|c| }
\hline
\bf Cs-a-1 & \bf A & \bf B & \bf C & \bf D & \bf E & \bf F & G & H & I & I\textsubscript{0} & I\textsubscript{1} & J & K & \bf L \\
\hline
\hline
Step 1 & & & D' & 1 & & & & D' & & & & & & \\
\hline
Step 2 & & & D' & 1 & 0 & & & D' & D' & D' & D' & & & \\
\hline
Step 3 & & & D' & 1 & 0 & 0 & & D' & D' & D' & D' & & D' & \\
\hline
Step 4 & & & D' & 1 & 0 & 0 & & D' & D' & D' & D' & 0 & D' & D' \\
\hline
Step 5 & & & D' & 1 & 0 & 0 & 1 & D' & D' & D' & D' & 0 & D' & D' \\
\hline
Step 6 & 1 & 1 & D' & 1 & 0 & 0 & 1 & D' & D' & D' & D' & 0 & D' & D' \\
\hline
\end{tabular}



\begin{tabular}{ |c||c|c|c|c|c|c|c|c|c|c|c|c|c|c| }
\hline
\bf Fs-a-0 & \bf A & \bf B & \bf C & \bf D & \bf E & \bf F & G & H & I & I\textsubscript{0} & I\textsubscript{1} & J & K & \bf L \\
\hline
\hline
Step 1 & & & & & & D & & & & & 0 & & D & \\
\hline
Step 2 & & & & & & D & & & 0 & 0 & 0 & 0 & D & D \\
\hline
Step 3 & & & & & & D & 1 & & 0 & 0 & 0 & 0 & D & D \\
\hline
Step 4 & 1 & 1 & & & & D & 1 & & 0 & 0 & 0 & 0 & D & D \\
\hline
Step 5 & 1 & 1 & & & 0 & D & 1 & 0 & 0 & 0 & 0 & 0 & D & D \\
\hline
Step 6 & 1 & 1 & X & 0 & 0 & D & 1 & 0 & 0 & 0 & 0 & 0 & D & D \\
\hline
\end{tabular}

\subsection*{List of Detectable Faults}
\begin{tabular}{|c||c|c|c|c|c|c|}
    \hline
    Fault & A & B & C & D & E & F
    \\
    \hline
    \hline
    As-a-0 & D & 1 & X & 0 & 0 & 0 \\
    \hline
    As-a-1 & D' & 1 & X & 0 & 0 & 0 \\
    \hline
    Ds-a-1 & 1 & 1 & 1 & D' & 0 & 0 \\
    \hline
    Es-a-0 & 1 & 1 & X & 0 & D & 0 \\
    \hline
    I\textsubscript{1}s-a-0 & X & X & 1 & 1 & 0 & 0 \\
    \hline
    Bs-a-1 & 1 & D' & X & 0 & 0 & 0 \\
    \hline
    Cs-a-0 & 1 & 1 & D & 1 & 0 & 0 \\
    \hline
    Cs-a-1 & 1 & 1 & D' & 1 & 0 & 0 \\
    \hline
    Fs-a-0 & 1 & 1 & X & 0 & 0 & D \\
    \hline
\end{tabular}



\section*{3.1.4 Merging Test Patterns}
\begin{tabular}{|c||c|c|c|c|c|c|}
\hline
Fault & A & B & C & D & E & F \\
\hline
\hline
\rowcolor{green!50}As-a-0 & 1 & 1 & X & 0 & 0 & 0 \\
\hline
As-a-1 & 0 & 1 & X & 0 & 0 & 0 \\
\hline
\rowcolor{green!50}Ds-a-1 & 1 & 1 & 1 & 0 & 0 & 0 \\
\hline
Es-a-0 & 1 & 1 & X & 0 & 1 & 0 \\
\hline
\rowcolor{yellow!50}I1s-a-0 & X & X & 1 & 1 & 0 & 0 \\
\hline
Bs-a-1 & 1 & 0 & X & 0 & 0 & 0 \\
\hline
\rowcolor{yellow!50}Cs-a-0 & 1 & 1 & 1 & 1 & 0 & 0 \\
\hline
Cs-a-1 & 1 & 1 & 0 & 1 & 0 & 0 \\
\hline
Fs-a-0 & 1 & 1 & X & 0 & 0 & 1 \\
\hline
\end{tabular}

By considering `D' to be `1' and `D'' to be 0, we can merge the two rows highlighted green together, and the two rows highlighted yellow together.


\section*{3.1.5 List of Reduced Test Patterns}
From the previous step, we can reduce the test patterns to 7 tests:

\begin{tabular}{|c||c|c|c|c|c|c|c|}
\hline
Test & A & B & C & D & E & F & L \\
\hline
\hline
Ds-a-1 & 1 & 1 & 1 & 0 & 0 & 0 & 0 \\
\hline
As-a-1 & 0 & 1 & 0 & 0 & 0 & 0 & 1 \\
\hline
Cs-a-0 & 1 & 1 & 1 & 1 & 0 & 0 & 1 \\
\hline
Es-a-0 & 1 & 1 & 0 & 0 & 1 & 0 & 1 \\
\hline
Bs-a-1 & 1 & 0 & 0 & 0 & 0 & 0 & 1 \\
\hline
Cs-a-1 & 1 & 1 & 0 & 1 & 0 & 0 & 0 \\
\hline
Fs-a-0 & 1 & 1 & 0 & 0 & 0 & 1 & 1 \\
\hline
\end{tabular}


\section*{3.1.6 Fault Simulation}
\begin{tabular}{|c||c|c|c|c|c|c|c|c|c|c|c|c|c|c|}
\hline
Pattern & A & B & C & D & E & F & G & H & I & I\textsubscript{0} & I\textsubscript{1} & J & K & L \\
\hline
\hline
As-a-1 & s-a-1 & & & & & s-a-1 & s-a-1 & & & s-a-1 & s-a-1 & s-a-0 & s-a-1 & s-a-1 \\
\hline
Bs-a-1 & & s-a-1 & & & & & s-a-1 & & & s-a-1 & & s-a-0 & & s-a-0 \\
\hline
Cs-a-0 & & & s-a-0 & s-a-0 & & & & s-a-0 & s-a-0 & & & & & \\
\hline
Cs-a-1 & & & s-a-1 & & s-a-1 & & & s-a-1 & s-a-1 & & & & & \\
\hline
Ds-a-1 & & & & s-a-1 & s-a-1 & & & s-a-1 & s-a-1 & & & & & \\
\hline
Es-a-0 & & & & & s-a-0 & & & & s-a-0 & & & & & \\
\hline
Fs-a-0 & & & & & & s-a-0 & & & & & & & s-a-0 & s-a-0 \\
\hline
\end{tabular}



\chapter*{Task B: Built-In Self-Test (BIST)}
\section*{3.2.1 Memory Content File}

\end{document}
