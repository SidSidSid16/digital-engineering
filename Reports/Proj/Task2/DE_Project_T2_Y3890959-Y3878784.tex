\documentclass[11pt]{report}
\usepackage[utf8]{inputenc}
\usepackage[margin=2.0cm]{geometry}
\usepackage{tikz}
\usepackage{fancyhdr}
\usepackage{xcolor}
\usepackage{minted}
\usepackage{graphicx}
\usepackage[parfill]{parskip}
\usepackage{lscape}
\usepackage{multirow}

\usetikzlibrary{automata, positioning, arrows, fit}

\title{Digital Engineering\\Project Task 1}
\author{Y3890959\\Y3878784}
\date{28th February 2023}

\pagestyle{fancy}
\fancyhead{}
\setlength{\headheight}{14pt}
\fancyhead[L]{Project Task 1}
\fancyhead[R]{Y3890959, Y3878784}
\fancyfoot{}
\fancyfoot[L]{Digital Engineering}
\fancyfoot[R]{\thepage}

\makeatletter
\let\ps@plain\ps@fancy 
\makeatother

\setminted {
    fontsize=\footnotesize,
    frame=single,
}

\begin{document}

\maketitle

\chapter*{Task 2: Clock domain crossing using dual-clock FIFOs}

\section*{1 - Description of FSMs}

\subsection*{SOURCE\_CTRL FSM}

\begin{figure}[h] % ’ht’ tells LaTeX to place the figure ’here’ or at the top of the page
    \centering % centers the figure
    \begin{tikzpicture}[->,>=stealth',shorten >=1pt,auto,node distance=7cm,semithick]
        \tikzstyle{every state}=[draw=none,text=black]

        \node[initial,state] (A)              {\textbf{IDLE}};;
        \node[state]         (B) [right of=A] {\textbf{COMP}};
        \node[state]         (C) [right of=B] {\textbf{HOLD}};

        \path (A) edge                         node[align=center, scale=0.8] {FROM\_OUTPUT = 0\\en = 1} (B)
              (B) edge [above, bend left=50]  node[align=center, scale=0.8] {FIFO\_FULL = 1} (C)
                  edge [above, bend right=50]  node[align=center, scale=0.8] {SWITCHES = COUNTER\\(or rst = 1)} (A)
              (C) edge                         node[align=center, scale=0.8] {FIFO\_FULL = 0} (B)
                  edge [above, bend left=50]   node[scale=0.8] {rst = 1} (A);
    \end{tikzpicture}
    \caption{SOURCE\_CTRL FSM state graph}
    \label{fig:my_label}
\end{figure}

The FSM starts at IDLE state, where all internal signals will reset and the LEDs are off. When the `en'
pushbutton is pressed and the output logic FSM has completed (FROM\_SOURCE goes low) the state goes to COMP
where the outputs are computed. As soon as `en' is toggled, when the FSM is still in IDLE, the values of the
SWITCHES is stored in a register `LIMT\_REG'. In the COMP state, the `LIMT\_CNT' is enabled and counts,
`EN\_SOURCE' also goes high so the outputs are computed, and the `FIFO\_WR\_EN' goes high so the values are
stored.

\end{document}
